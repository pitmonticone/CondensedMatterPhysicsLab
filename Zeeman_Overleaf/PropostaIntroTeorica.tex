\documentclass[a4paper,12pt,abstracton]{scrartcl}
\usepackage[utf8]{inputenc}
\usepackage{float}
\usepackage{amsmath}
\usepackage{amssymb}
\usepackage{pifont}% http://ctan.org/pkg/pifont
\usepackage[font=small,labelfont=bf]{caption}
\usepackage{graphicx}
%\usepackage{dirtytalk}
\usepackage{multicol}
\usepackage{booktabs}
\usepackage{colortbl}
\usepackage{appendix}
\usepackage{nomencl}
\usepackage{lmodern}
\usepackage[nottoc]{tocbibind}
\usepackage{xcolor}
%\graphicspath{images/}
\usepackage[margin = 2cm]{geometry}
\usepackage{ragged2e} % good alignment
\usepackage{hyperref}
\usepackage{siunitx} % Provides the \SI{}{} and \si{} command for typesetting SI units
\hypersetup{colorlinks=true,
    linkcolor=blue,
    filecolor=magenta,      
    urlcolor=cyan, 
    citecolor=gray}

%\DeclareGraphicsExtensions{.png,.pdf} % low-res (work in progress)
%\DeclareGraphicsExtensions{.pdf,.png}  % high-res (final draft)
%\setlength\parindent{0pt} % Removes all indentation from paragraphs
%\bibliographystyle{unstr}
\setlength\parindent{0pt}
\setlength{\parskip}{0.3em}
\newcommand{\xmark}{\ding{55}}

\renewcommand{\nomname}{List of Symbols}
\renewcommand{\nompreamble}{The following list explains the symbols used within the body of the report.}

\usepackage{etoolbox}
\renewcommand\nomgroup[1]{%
  \item[\bfseries
  \ifstrequal{#1}{E}{Experimental Equipment}{%
  \ifstrequal{#1}{C}{Computational Methods}{%
  \ifstrequal{#1}{T}{Theoretical Concepts}{
  \ifstrequal{#1}{P}{Physical Constants}{}}}}%
]}


\subject{CMP Lab Report} % Matter Physics, Physics, Chemical Physics ?
\title{Zeeman Effect}
\author{Group B9\footnote{Pietro Monticone , Claudio Moroni , Alberto Mosso , Riccardo Valperga.}}

\renewcommand{\listfigurename}{Plots}
\renewcommand{\listtablename}{Tables}
\renewcommand{\nomname}{Nomenclature}


\begin{document}
\maketitle
\makenomenclature
\begin{abstract}
something..
\end{abstract}
\clearpage
\tableofcontents
%\listoffigures
%\listoftables

\mbox{
\nomenclature[C]{MC}{Monte Carlo}
\nomenclature[E]{CB}{Cosmic Box}
\nomenclature[P]{$m_e$}{Electronic Mass}
\nomenclature[P]{$e$}{Electronic Charge}
\nomenclature[T]{$V_{disc}$}{Discriminator Voltage}
}
\newpage
\setlength{\columnsep}{27pt} 
\begin{multicols}{2}
\printnomenclature
\end{multicols}
\newpage
\section{Theoretical Introduction}
Zeemann effect is a phenomenon where the submission of a spectroscopic light source to a magnetic fields causes the splitting of the observed spectral lines. Because a spectral line is associated to an atomic energy level, such splitting is always related to a degeneracy breaking. In an atom, degeneracy is due to the orbital momentum l, its third component $m_l$, and the z-component of spin $m_s$ (s is fixed  because electrons are fermions, so $s=\frac{1}{2}$.\newline
Normal Zeemann effect is observed when the total spins of the base and target level of the transition are zero.
Anomalous Zeeman effect is instead detected when the spins of the base and target level of the transition are not zero.
\subsection{Normal Zeemann effect}
An external magnetic field $\vec{B}$ gives an energy contribution to each level
\begin{equation}
\Delta E = - \vec{\mu} \cdot \vec{B}
\end{equation}
Where $\vec{\mu}$ is the magnetic moment. We consider $\vec{B}$ parallel to the z-axis, so we are only interested in $\mu_z$. One finds: 
\begin{equation}
mu_z \equiv -\frac{e}{2m}L_z = -\frac{e}{2m}\hbar m_lg_l
\end{equation}
Where $g_l$ is a constant equal to 1.Thus:
\begin{equation}
\Delta E = - \mu_z B = \frac{e \hbar}{2m} m_l B = \mu_B m_l B
\end{equation}
Where $\mu_B$ is the Bohr's magneton, whose theoretical value is
$\mu_B = $. Because $\Delta E$ depends on $m_l$, degeneracy in $m_l$ is broken and thus spectral lines split.
\subsection{Anomalous Zeemann Effect}
When spin is involved, the expression for $\vec{\mu}$ to use is 
\begin{equation}
\vec{\mu}=-\frac{eg_l}{2m} \vec{L}_J - \frac{eg_s}{2m} \vec{S}_J
\end{equation}
Where $\vec{L}_J$ and $ \vec{S}_J$ are the projections of $\vec{L}$ and $\vec{S}$ along the total angular momentum $\vec{J}:= \vec{L}+ \vec{S}$, and $g_s$ is about 2. One finds:
\begin{equation}
\vec{\mu} = -\frac{e}{2m}g_{jls}\vec{J}
\end{equation}
Where 
\begin{equation}
g_{jls}=\frac{3j(j+1)-l(l+1)+s(s+1)}{2j(j+1)}
\end{equation}
Pointing $\vec{B}$ along the z-axis, we have $\mu_z =-\frac{e \hbar}{2m}g_{jls}m_j=-\mu_B g_{jls}m_j$, thus
\begin{equation}
\Delta E = -\vec{\mu}\cdot \vec{B}= \mu_z B = \mu_Bg_{jls}m_j B
\end{equation}
When considering anomalous Zeemann effect, both spinorial and orbital degeneracy are broken, giving rise to even more splitting.


%\bibliographystyle{unsrt}
%\bibliography{biblio}
\end{document}
