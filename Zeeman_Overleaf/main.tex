\documentclass[a4paper,12pt,abstracton]{scrartcl}
\usepackage[utf8]{inputenc}
\usepackage{float}
\usepackage{amsmath}
\usepackage{amssymb}
\usepackage{pifont}% http://ctan.org/pkg/pifont
\usepackage[font=small,labelfont=bf]{caption}
\usepackage{graphicx}
\usepackage{dirtytalk}
\usepackage{multicol}
\usepackage{booktabs}
\usepackage{colortbl}
\usepackage{appendix}
\usepackage{nomencl}
\usepackage{lmodern}
\usepackage[nottoc]{tocbibind}
\usepackage{xcolor}
%\graphicspath{images/}
\usepackage[margin = 3cm]{geometry}
\usepackage{ragged2e} % good alignment
\usepackage{hyperref}
\usepackage{siunitx} % Provides the \SI{}{} and \si{} command for typesetting SI units
\hypersetup{colorlinks=true,
    linkcolor=blue,
    filecolor=magenta,      
    urlcolor=cyan, 
    citecolor=gray}

%\DeclareGraphicsExtensions{.png,.pdf} % low-res (work in progress)
%\DeclareGraphicsExtensions{.pdf,.png}  % high-res (final draft)
%\setlength\parindent{0pt} % Removes all indentation from paragraphs
%\bibliographystyle{unstr}
\setlength\parindent{0pt}
\setlength{\parskip}{0.3em}
\newcommand{\xmark}{\ding{55}}

\renewcommand{\nomname}{List of Symbols}
\renewcommand{\nompreamble}{The following list explains the symbols used within the body of the report.}

\usepackage{etoolbox}
\renewcommand\nomgroup[1]{%
  \item[\bfseries
  \ifstrequal{#1}{E}{Experimental Equipment}{%
  \ifstrequal{#1}{C}{Computational Methods}{%
  \ifstrequal{#1}{T}{Theoretical Concepts}{
  \ifstrequal{#1}{P}{Physical Constants}{}}}}%
]}


\subject{CMP Lab Report} % Matter Physics, Physics, Chemical Physics ?
\title{Zeeman Effect}
\author{Group B9\footnote{Pietro Monticone , Claudio Moroni , Alberto Mosso , Riccardo Valperga.}}

\renewcommand{\listfigurename}{Plots}
\renewcommand{\listtablename}{Tables}
\renewcommand{\nomname}{Nomenclature}


\begin{document}
\maketitle
\makenomenclature
\begin{abstract}
This experiment examined the normal and anomalous Zeeman effect with the aim of calculating the Bohr magneton by fitting the energy gap between spectral lines of a cadmium lampas surrounded in a magnetic field, as a function of the magnitude of the magnetic field. In the first section, the magnet calibration and the procedure adopted to test the uniformity of the magnetic field, are described. Regarding the former, we expected the magnetic field to vary almost linearly with the current since we worked with soft ferromagnets. For what concern the latter, we have made sure that for small movement in the region occupied by the cadmium lamp, the magnitude of the magnetic field did not vary substantially. Lastly, with a CCD camera and with the help of a polarizing filters and a retarding lamina, several images of the splitted spectral lines have been taken in several different configurations. The configurations depended on the longitudinal or traversal orientation of the magnets compared to the optical axe of the experimental apparatus; and the different configuration of the polarizing filters placed along the light path. The results are four estimates of the Bohr magneton that has been compared with the theoretical value.
\end{abstract}
\clearpage
\tableofcontents
%\listoffigures
%\listoftables

\mbox{
\nomenclature[C]{MC}{Monte Carlo}
\nomenclature[E]{CB}{Cosmic Box}
\nomenclature[P]{$m_e$}{Electronic Mass}
\nomenclature[P]{$e$}{Electronic Charge}
\nomenclature[T]{$V_{disc}$}{Discriminator Voltage}
}
\newpage
\setlength{\columnsep}{27pt} 
\begin{multicols}{2}
\printnomenclature
\end{multicols}
\newpage


%\bibliographystyle{unsrt}
%\bibliography{biblio}
\end{document}
