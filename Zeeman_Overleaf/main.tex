\documentclass[a4paper,12pt,abstracton]{scrartcl}
\usepackage[utf8]{inputenc}
\usepackage{float}
\usepackage{tikz}
\usepackage{amsmath}
\usepackage{amssymb}
\usepackage{pifont}% http://ctan.org/pkg/pifont
\usepackage[font=small,labelfont=bf]{caption}
\usepackage{graphicx}
\usepackage{dirtytalk}
\usepackage{multicol}
\usepackage{booktabs}
\usepackage{colortbl}
\usepackage{appendix}
\usepackage{nomencl}
\usepackage{lmodern}
\usepackage[nottoc]{tocbibind}
\usepackage{xcolor}
%\graphicspath{images/}
\usepackage[margin = 3cm]{geometry}
\usepackage{ragged2e} % good alignment
\usepackage{hyperref}
\usepackage{siunitx} % Provides the \SI{}{} and \si{} command for typesetting SI units
\hypersetup{colorlinks=true,
    linkcolor=blue,
    filecolor=magenta,      
    urlcolor=cyan, 
    citecolor=gray}

%\DeclareGraphicsExtensions{.png,.pdf} % low-res (work in progress)
%\DeclareGraphicsExtensions{.pdf,.png}  % high-res (final draft)
%\setlength\parindent{0pt} % Removes all indentation from paragraphs
%\bibliographystyle{unstr}
\setlength\parindent{0pt}
\setlength{\parskip}{0.3em}
\newcommand{\xmark}{\ding{55}}

\renewcommand{\nomname}{List of Symbols}
\renewcommand{\nompreamble}{The following list explains the symbols used within the body of the report.}

\usepackage{etoolbox}
\renewcommand\nomgroup[1]{%
  \item[\bfseries
  \ifstrequal{#1}{E}{Experimental Equipment}{%
  \ifstrequal{#1}{C}{Computational Methods}{%
  \ifstrequal{#1}{T}{Theoretical Concepts}{
  \ifstrequal{#1}{P}{Physical Constants}{}}}}%
]}


\subject{CMP Lab Report} % Matter Physics, Physics, Chemical Physics ?
\title{Zeeman Effect}
\author{Group B9\footnote{Pietro Monticone , Claudio Moroni , Alberto Mosso , Riccardo Valperga.}}

\renewcommand{\listfigurename}{Plots}
\renewcommand{\listtablename}{Tables}
\renewcommand{\nomname}{Nomenclature}


\begin{document}
\maketitle
\makenomenclature
\begin{abstract}
This experiment examined the normal and anomalous Zeeman effect with the aim of calculating the Bohr magneton by fitting the energy gap between spectral lines of a cadmium lamp immersed in a magnetic field, as a function of its magnitude. The first section concerns the magnet calibration and the procedure we adopted to test the magnetic field homogeneity. As far as the former is concerned, we expected the magnetic field to vary almost linearly with the current since we worked with soft ferromagnets. Regarding the latter, we have made sure that for small displacements in the region occupied by the cadmium lamp, the magnitude of the magnetic field did not vary substantially. Lastly, after the global experimental apparatus has been endowed with polarizing filters, a CCD camera and a retarding lamina, several images of the splitted spectral lines have been taken in various states depending from the chosen orientation (longitudinal or traversal) of the magnets with respect to the optical axe and the configuration of the polarizing filters placed along the light path. The results consist of four estimates of the Bohr magneton and the relevant compatibility tests .
\end{abstract}
\clearpage
\tableofcontents
%\listoffigures
%\listoftables

\mbox{
\nomenclature[C]{MC}{Monte Carlo}
\nomenclature[E]{CB}{Cosmic Box}
\nomenclature[P]{$m_e$}{Electronic Mass}
\nomenclature[P]{$e$}{Electronic Charge}
\nomenclature[T]{$V_{disc}$}{Discriminator Voltage}
}
\newpage
\setlength{\columnsep}{27pt} 
\begin{multicols}{2}
\printnomenclature
\end{multicols}
\newpage

\section{Introduction}
\subsection{Theoretical Introduction}
\subsection{Experimental Introduction}

\section{Calibration Curve}
In order to calibrate the magnet for the future measures, we determined the dependence B(I) of the magnetic field from the current injected in the coils. With the use of the magnetometer, in the way already described in the previous section, we obtained four calibration curves that have been reported below. Table 1 and Table 3 report the data obtained increasing the current ($0A \longrightarrow 10A$) and the other two, by decreasing it ($10A \longrightarrow 0A$). 

%tabelle 1 2 3 4 secondo me due per riga, per non sprecare troppo spazio e non sembrare noiosi, non sarebbe male!


Although the magnet was a soft magnet, for high current value, it began to respond in a non-linear way. Therefore, a linear fit has been tried for the first points of the datasets (red line), and a parabolic fit has been tried for the last ones (red line). Below, the corresponding plots have been reported: Figure 1 and 3 refer to the data collection while increasing the whereas Figure 2 and 4 refer to the other one.

%grafici fit rette/parabole 

The final curve, from which we took the several magnitudes of the magnetic field further on, has been evaluated in the following way: we averaged the four values of each parameter found, than we tested the compatibility of every single with the overall average. The final result, used as parameter for the final curve, is a new average that has been evaluated with just the parameters that resulted compatible with the overall average. The compatibility tests and the plot of the final curve are reported below in Table ? and Figure ?? respectively.

%tabelle test Z e grafico con curva finale

***possibile conclusion per questa parte***
we observed a residual magnetization of about 6 mT of the magnets in absence of current in the coils. This residual magnetization could have been the cause of the incompatibility of the overall average and the single parameters. despite the residual magnetization, that seemed to disappear from one day to another, this seemed the best method to estimate the calibration curve.

\section{Magnetic Field Homogeneity}


\section{Normal Zeeman Effect}

\subsection{Data Collection}


Below, in table 1 and table 2, the data regarding the trasversal and longitudinal configurations are reported. The values of the currents are the ones from which the values of B have been calculated by using the calibration curve. the ***d/D*** and ***sd/D*** values are the averages and the standard deviations calculated as described in Section XX.
            %tabella trasversale e longitudinale
            
\subsection{Data Analysis \& Visualization}

The datasets have been fit with a liner trend: Figure X shows the plot of the transversal configuration while Figure Y shows the other.

            %grafici rasversale e longitudinale
            

\section{Anomalous Zeeman Effect}


\subsection{Data Collection}
%template sentences
The dataset has been collected and reported in \hyperref[table:boh]{Table \ref*{table:boh}}. \newline 

The measures has been collected and $X$ values have been calculates as displayed in \hyperref[table:znl]{Table \ref*{table:znl}}.

\subsection{Data Analysis \& Visualization}
The dataset has been fit with the function $$\epsilon(HV)=\frac{\boldsymbol{p_0}}{1+e^{\frac{\boldsymbol{p_1}-HV}{\boldsymbol{p_2}}}}$$
and plotted in \hyperref[fig:e(V)fal]{Figure \ref*{fig:e(V)fal}} and \hyperref[fig:e(V)fgl]{ \ref*{fig:e(V)fgl}}  together with the fit parameters $\boldsymbol{p_0}$, $\boldsymbol{p_1}$, $\boldsymbol{p_2}$ .

\section{Conclusions}


%\bibliographystyle{unsrt}
%\bibliography{biblio}
\end{document}
